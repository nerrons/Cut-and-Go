\documentclass[10pt]{article}

\usepackage[a4paper,
	top=28mm, right=60mm, bottom=40mm, left=20mm]{geometry}

\usepackage{fontspec, xeCJK}
\setmainfont[Ligatures={Rare, Historic}]{EB Garamond}
\setCJKmainfont{Source Han Serif TC}

\setlength{\parindent}{0em}
\setlength{\parskip}{0.8em}
\linespread{1.3}

\usepackage{xcolor}
\definecolor{toto}{HTML}{E8D08F}

\usepackage{hyperref}
\hypersetup{
    colorlinks=true,
    linkcolor=gray,
    urlcolor=gray
}

\providecommand{\tightlist}{\setlength{\itemsep}{0pt}\setlength{\parskip}{0pt}}

\begin{document}
\begin{flushleft}
\textbf{分級與邊境漩渦}\\
Content Rating for Battlefield\\
\end{flushleft}
\vspace{3em}

村上春樹的英文譯者 \href{https://en.wikipedia.org/wiki/Jay_Rubin}{Jay
Rubin}
\href{https://book.douban.com/subject/34893737/}{講}日本戰後審查的表徵是「權力喪失恐懼症」,它「恰恰證明瞭當權者承認自己的權力有可能單憑語言和思想就能被顛覆」。然而當審查成為基準與慣例,這種恐懼就轉嫁給陰影下的創作者與希望跳脫審查去體驗各色奇狀物的人士{——}以國家認同感對他們進行絞殺,卻混淆為優勝劣汰。而分級固然是審查的一部分,在強烈審查之下的分級何等虛偽,試想頭腦固化的成年人身上有多少
filter,進階飼料只會讓其無法消化,自幼對藝文的不屑態度也使其難以反芻。

我更加關心時常被忽略而又滿重要的分級行為,今天再不冗談審查。換言之,不談「能否」而論「如果能,該否」。就影視來講,中國大陸和朝鮮皆缺乏電影分級制度。我對立法一點也不關心,由此假設年齡界定是動態的,標準是心理年齡而非法定年齡;同時分級方法也是動態的:相比於制定條款,經當下普世智識確認的分級更為重要。審查下現狀是,對優質電影慘遭刪節或拒於門外產生厭惡的朋友,可行選項有:要麼飛去肉空間或賽博空閒的他鄉(一直以來有去香港觀看完整電影的風尚,現在此舉似乎變得更困難),要麼冒風險讓實體碟越洋入內,要麼咬牙切齒地下載盜版。無論哪一種,似乎都少有人在意別處的分級現狀。

在高壓審查的國家,缺乏分級意味著平均化尤其是非意識形態衝突、關於性與暴力的內容,以普適所有人的口味。這些與年齡、身體相關的議題在肉空間缺乏討論,以致產生肉體的無所適從、風度盡失。之前恰在獅城看了去年金馬提名電影\href{https://zh.wikipedia.org/wiki/\%E7\%86\%B1\%E5\%B8\%B6\%E9\%9B\%A8}{《熱帶雨》}\footnote{別註:我一直想展示其中段落:校管弦樂團行進橫在男女主角之間,阻隔他們對話。讓我想起《牯嶺街》裡一幕,同樣是學校裡,管弦樂聲為男女主角的對峙提供間隙。如今您欲看《熱帶雨》,惟二之一的辦法是成為東南亞影視串流平台
HOOQ 的訂戶。(更新:HOOQ
服務於今年四月卅日\href{https://hooq.tv}{終結}。)}。那已是二次上映,排期很少,看的人不多。Golden
Village
售票員姊姊望向小隻的我,笑著特地在電影票上圈出「18」。我方才過完十九歲生日,不過那晚是第一次在多人場域看非典型性愛畫面。身旁坐有一對年輕夫婦,不知是否嘀咕著小孩打擾了他們的雅緻。如果從未感受在公共空間討論性愛關係或血腥行為的氛圍,對於它們的羞恥心和恐懼感會使人趨向病態。因為無法確認世人的反應,也難以得出健康的結論。同時溫和題材對觀者亦是種馴化,很大程度上延長了一個人的幼稚階段。

分級行為對於活在代際分歧漩渦中的朋友來說,或許能夠提供某種引導。時常講我的生父不是我的養父,矽谷與日本文化才是。他們都不稱職,後者一方面是因為在本源社會與原生家庭教育匱乏的環境下,他們無法提供足夠系統性的資源而僅被大量地擷取局部,從而進行符號化的習作,妳有在期待著什麼嗎?妳對自我的定力絕對坦誠嗎?而另一方面,越級的觀閱行為會讓人過於集中於刺激性段落,而猛吸其中反叛與官能元素。還在唸中學的某個週六我在學校電腦上看\href{https://en.wikipedia.org/wiki/Sh\%C5\%8Djo_Tsubaki}{《少女椿》}。對分級無意識、爛漫野蠻的雜食行為默然間訓練出可笑又可恥的
ego,亂而無序的認知會撕裂人心。一種外顯是,憤怒、仇恨與哀愁漫過快樂,而快樂是離散、波動的且在特定極端的地方劇烈燃燒。證明某樣東西有毒素與排毒是很花時間的。

我明白比起定力,普羅大眾更缺乏的是 rebellious
的精魂。但在各種文化的邊境經歷激盪的朋友,卻是罹難而羸弱。他們被強力牽引,發出者卻是垂簾聽政、安享暮年而將逐漸喪失肌肉壓迫之人士。分級行為能夠培養定力,同時指向未來,是下一代的美育。



\vspace{3em}
\begin{flushleft}
\small{御薬袋托托\\
二〇二〇年三月廿四日}
\end{flushleft}
\end{document}