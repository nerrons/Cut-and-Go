\documentclass[10pt]{article}

\usepackage[a4paper,
	top=28mm, right=60mm, bottom=40mm, left=20mm]{geometry}

\usepackage{fontspec, xeCJK}
\setmainfont[Ligatures={Rare, Historic}]{EB Garamond}
\setCJKmainfont{Source Han Serif TC}

\setlength{\parindent}{0em}
\setlength{\parskip}{0.8em}
\linespread{1.3}

\usepackage{xcolor}
\definecolor{toto}{HTML}{E8D08F}

\usepackage{hyperref}
\hypersetup{
    colorlinks=true,     
    urlcolor=gray
}

\providecommand{\tightlist}{\setlength{\itemsep}{0pt}\setlength{\parskip}{0pt}}

\begin{document}
\begin{flushleft}
\textbf{殺死純愛}\\
Kill Love\\
\end{flushleft}
\vspace{3em}

歲末楊德昌的兩部電影《牯嶺街少年殺人事件》《一一》在網飛上提供,我觀看了這兩部分別映於九一、〇〇年的電影。在裡面可以看到如今功成名就的臺灣電影人的身影(《牯嶺街少年殺人事件》可以看到陳以文、湯湘竹)。同時不得不說,臺語、國語、英文和日文交織在一起,是楊德昌電影裡面可愛的一處。在電影裡我聽到一些好玩的音樂。在《牯嶺街少年殺人事件》中,除了滿滿是貓王經典的情歌外,有一個橋段是小貓王墊上木箱,用假聲唱〈Angel
Baby〉,背景設有掛著五色的小燈泡,美國和臺灣的旗子,灑滿紫綠交織、柔和而懶散的舞臺燈,後面畫的有海灘。讓我忽然憶起戶川純在八五年「好きよ、好き」的神經質版本。我第一口咬下去的是戶川純,那是在視野模糊了的深夜,沒有意識到這首純愛曲可溯到六〇年,原唱是
\href{https://www.youtube.com/watch?v=zQRYk7stYIw}{Rosie and the
Originals}。

歌詞被戶川純從「Please never leave me / Blue and alone / If you ever go
/ I'm sure you'll come back home / \textbf{Because I love you} /
\textbf{I love you, I do}」更改為「Please never leave me って言ったら /
私の手を握ってくれた / \textbf{He said I love you, I love you, I
do}」。請求對方不要走的角色,從女孩子移位到男孩子,宜恰地描繪了戶川純對純愛的體會。這樣女性主導的純愛慾念,或多或少地融入了我的血液和骨頭裡面。此種慾念也可以在其他地方找到,比如說戶川純有寫過一篇叫\href{https://juntogawa.org/something-extra}{《愚行》}的
novelette{——}情變後提出分手的男子收到舊愛的食指作為離別禮,之後怒火攻心的女子實施報復、拷問純愛真諦、用犧牲換取和解。戶川純趨及的是無可救藥、不惜代價的血紅色的女王純愛。

楊德昌聽過戶川純嗎?我知道\href{https://www.zhihu.com/question/364129176}{問這種問題}會很無聊。小四兒無法理解戀人對世界固定態的悲觀,信仰與戀愛的衝突促成了殺意。這種悲觀主義的重壓源於代際間的混沌,是傳承與叛逆的綜合體,時不時縈繞在身邊的無力感轉化為一種對現實的障礙和憎惡,逼迫著我們殺死那些珍愛的東西。凡此種種的姿態,今猶存愈烈。但純愛絕對不是解藥,特別是通過扭曲對方的未來而為對方著想的純愛。別忘了小四兒行凶的刀用的是日本女人用來自殺的刀。

《一一》裡面楊德昌的第二任妻子彭鎧立彈了幾首巴赫、貝多芬和貝利尼的曲子。我還記得是其中在東京的酒吧裡面,大田彈唱過{坂}本九的\href{https://de.wikipedia.org/wiki/Sukiyaki_(Lied)}{〈上を向いて{歩}こう〉}(\emph{Sukiyaki},一九六一年)。其他的不太記得啦。

\vspace{3em}
\begin{flushleft}
\small{御薬袋托托\\
二〇二〇年六月十七日}
\end{flushleft}
\end{document}