\documentclass[10pt]{article}

\usepackage[a4paper,
	top=28mm, right=60mm, bottom=40mm, left=20mm]{geometry}

\usepackage{fontspec, xeCJK}
\setmainfont[Ligatures={Rare, Historic}]{EB Garamond}
\setCJKmainfont{Source Han Serif TC}

\setlength{\parindent}{0em}
\setlength{\parskip}{0.8em}
\linespread{1.3}

\usepackage{xcolor}
\definecolor{toto}{HTML}{E8D08F}

\usepackage{hyperref}
\hypersetup{
    colorlinks=true,     
    urlcolor=gray
}

\providecommand{\tightlist}{\setlength{\itemsep}{0pt}\setlength{\parskip}{0pt}}

\begin{document}
\begin{flushleft}
\textbf{青年側記}\\
Years of Rebellious Continuing\\
\end{flushleft}
\vspace{3em}

\textbf{識水性的水怪}

還在假惺惺地講包容的人,是在計劃一場屠殺。好意容納他見,實則強置共性。像是被困在一間從底部開始漲水的封閉房間。起初沒有任何人想過堵住源頭,積而成患。後來者,若不想溺水就得去學游泳。「為了活下來,跟著我一起學游泳吧?」資歷者笑道。妳嗆了幾口令人窒息的水,四肢開始劇烈擺動。於是學會了讓自己浮起來。

泳姿美嗎?有人在乎嗎?酥麻在一片渾水。有一天妳游累了抬頭看房頂。似乎還可笑的很遠。

「跟我學潛水⋯⋯」妳撥乾臉上的水珠,面前是一位說不清長相的人,第一反應是個要把妳拖下水的水怪。「混蛋,聽著。跟我學潛水吧,我們去關掉噴水的裝置。還不明白嗎?有一天我們會被水完全包圍。」

「不,你是惡魔嗎?還是頭腦不好使。現在的狀態滿好。」妳好討厭不切實際的人。何況末日不是現在該考慮的事情。水怪生氣地拍打水面,濺得妳滿臉水花:「膽敢口出此言。想像力被妳藏到哪裡去了?還是妳聽了太多太多老人的話{——}『未來與你何干』。他們的生命根本見不到審判,妳為何不考慮終身的事情、下一代的事情。跟我走吧⋯⋯」

將信將疑。猛吸一口氧氣,妳跟著識水性的怪獸前往底部。深藍的泡沫,渾水與浮屍⋯⋯結局會像是泡騰片嗎?不過這麼多年他們都
rebel 過來了。

\vspace{1em}
\textbf{豔陽下颳風}

小時候讀過莫泊桑寫的\href{http://www.online-literature.com/maupassant/4276/}{《A
Mother of
Monsters》}(一八八三),講述一位母親為了生出畸形的孩子用藤條束縛肚子的故事,讓我做了惡夢。究其原因是賣掉這些怪獸以饜足獵奇者的胃口。讓我做惡夢的部分是我自己不知怎地被吸進這位母親的羊水中,我呼吸著僅有的空氣,而自己正處於生長期的細胞感受到如同被大象之腳踩踏的壓迫感。這樣的環境成為了我世界的全部,也許無法知曉另外一種存活的方式。一是因為畸形的狀態使得我命不長矣目且短矣,另一是因為被觀賞的狀態而產生的「表演感」{——}被實驗與消費,但是卻
cannot tell 是展示多樣性的一環還是獨立的存在,或許皆否。

我的藤條是什麼呢?Toxic childhood
上一輩破碎的婚姻、本源與世界文化的斷層,還是腦部某些情感區域的基因陰霾。(不過這麼多年我都
rebel
過來了。)這種病徵愈漸清晰,而我越來越覺得與其怪罪於外界,不如討論治癒的可能。治療戒斷反應的「{薬}」究竟在哪裡呢?

\vspace{3em}
\begin{flushleft}
\small{御薬袋托托\\
二〇二〇年五月十二日}
\end{flushleft}
\end{document}