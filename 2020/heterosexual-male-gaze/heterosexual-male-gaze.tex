\documentclass[10pt]{article}

\usepackage[a4paper,
	top=28mm, right=60mm, bottom=40mm, left=20mm]{geometry}

\usepackage{fontspec, xeCJK}
\setmainfont[Ligatures={Rare, Historic}]{EB Garamond}
\setCJKmainfont{Source Han Serif TC}

\setlength{\parindent}{0em}
\setlength{\parskip}{0.8em}
\linespread{1.3}

\usepackage{xcolor}
\definecolor{toto}{HTML}{E8D08F}

\usepackage{hyperref}
\hypersetup{
    colorlinks=true,     
    urlcolor=gray
}

\providecommand{\tightlist}{\setlength{\itemsep}{0pt}\setlength{\parskip}{0pt}}

\begin{document}
\begin{flushleft}
\textbf{異男凝視}\\
Heterosexual Male Gaze\\
\end{flushleft}
\vspace{3em}

Luis Buñuel
滿腹苦楚,又哪會想到,既灌下兩杯馬天尼,片子就有了著落。嚼著最後一顆橄欖,酸澀汁液滑過舌頭。他扭過頭望向身旁的監製,半戲謔道:「嘿,Silberman。若慾念謊言混雜於同一女體,不如安排兩人來飾她。」那是在七七年某夜馬德里酒吧,弄得導演焦額的女主角處理方式就此敲定,成就了那年的荒誕劇情片{——}\emph{That
Obscure Object of Desire}。

古來好色男子多不勝數,事實上本片就改編自情色文學家 Pierre Louÿs
十九世紀末小說《The Woman and the Puppet》。此題數次被搬上螢幕,導演
Luis Buñuel
又借此換上現代框架,順帶諷刺當時的右翼運動。主體敘述巴黎好色男子 Mathieu
在從塞維亞返程途中向同車廂的人回憶情事:自己如何被迷不開身,又是如何被折磨背叛。去來反覆,終定決心與情人決裂。戲中的女主角
Conchita,一面是從西班牙遷來巴黎謀生、迷霧般羞赧又暗藏辛辣的 Carole
Bouquet;一面是熾烈的弗朗明哥舞女 Ángela
Molina,風情萬種,空中浮著她的異域色誘素。(開頭階段我還在暗自驚訝她們髮型不一樣的
goof。)兩個形象在轉場中更替、重疊,最終混淆觀眾的感官{——}每種特質都被放大,而這一游移的形象又讓人感到不安。

不禁讓人想起谷崎潤一郎二五年小說《痴人之愛》中的河合讓治。與 Mathieu
相比,同是對女體進行某種文化歸順:巴黎人 Mathieu 教 Conchita
識紅酒杯諸如此類、河合讓治對ナオミ一系列歐式馴化;Mathieu
去戀人故土製造偶遇、讓治尾隨ナオミ揭露情變謊言,全身上下流露出偽善的
stalker
味道;時不時就會偷跑出來的異男凝視;兩人同是深受嫉妒和過度相信之害,雖然怒火滿腔,卻依舊願望和解。他們在外人看來擁有一種天真的善意,而骨內的虛偽卻說服自己忽視這種天真,默認戀情的正當性(Mathieu
在公共空間講述遭遇,像是在宣佈他認可的正義)。這些事情今日依舊在發生。

\vspace{3em}
\begin{flushleft}
\small{御薬袋托托\\
二〇二〇年一月三日}
\end{flushleft}
\end{document}