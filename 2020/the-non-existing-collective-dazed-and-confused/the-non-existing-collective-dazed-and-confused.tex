\documentclass[10pt]{article}

\usepackage[a4paper,
	top=28mm, right=60mm, bottom=40mm, left=20mm]{geometry}

\usepackage{fontspec, xeCJK}
\setmainfont[Ligatures={Rare, Historic}]{EB Garamond}
\setCJKmainfont{Source Han Serif TC}

\setlength{\parindent}{0em}
\setlength{\parskip}{0.8em}
\linespread{1.3}

\usepackage{xcolor}
\definecolor{toto}{HTML}{E8D08F}

\usepackage{hyperref}
\hypersetup{
    colorlinks=true,     
    urlcolor=gray
}

\providecommand{\tightlist}{\setlength{\itemsep}{0pt}\setlength{\parskip}{0pt}}

\begin{document}
\begin{flushleft}
\textbf{無法集體頹廢}\\
The Non-Existing Collective Dazed and Confused\\
\end{flushleft}
\vspace{3em}

好後悔在十九歲才看 cult
片《\href{https://en.wikipedia.org/wiki/Dazed_and_Confused_(film)}{Dazed
and
Confused}》(\href{https://en.wikipedia.org/wiki/Richard_Linklater}{Richard
Linklater} 導,一九九三),不能夠成為我的童年。由一曲 Aerosmith
的《Sweet
Emotion》(一九七五)打頭,本片張幕於七六年德州高中畢業暑假的首日:留長髮聽搖滾樂團的少男少女競速破壞、霸凌整蠱、談情說愛。入夜,結伴行企在公路上,汽車後備箱冰鎮著喝不完的啤酒,在夜間派對幹架流血、哈
pot 泡 miss,汗液混著泥土,啤酒四濺(據說除了 Jason London
的角色都是在真飲)。他們攀電塔看夜裡城市全景,在深藍樹林裡同別的女孩擁吻。那些影片中盈滿情緒的畫面{——}早餐和屋前草坪上澀氣的吻別{——}像是一群青年本色演繹著自然派的叛逆。拍完後劇組中未成年姑娘
Milla Jovovich 就同男孩 elope 逃到了賭城。

原聲樂多到兩碟才灌完,電影預算的六分之一都拿去付了版權。其中有硬的
Foghat 的《Slow Ride》(一九七五)也有軟的 Seals and Crofts 的《Summer
Breeze》。恰到好處的舞廳曲、七三年 Dr. Smith
的騷靈藍調《\href{https://en.wikipedia.org/wiki/Right_Place,_Wrong_Time_(song)}{Right
Place Wrong Time}》(細野晴臣參與的樂隊
\href{https://ja.wikipedia.org/wiki/\%E3\%83\%95\%E3\%83\%AC\%E3\%83\%B3\%E3\%82\%BA\%E3\%83\%BB\%E3\%82\%AA\%E3\%83\%96\%E3\%83\%BB\%E3\%82\%A2\%E3\%83\%BC\%E3\%82\%B9}{Friends
of Earth} 在八六年的《Sex, Energy and
Star》中的同名樂取材於此歌)青澀迷人,似塗上了濛濛粉粉的糖霜。撥開背景曲的壁障,可觀得美國七〇風象。聽時腦海裡回憶著從少男淡紫色花紋襯衫裡流出的清朗的費洛蒙。

雖然這只抽離出來對美國廿世紀後期的世界印象與文化借鏡,但我們卻不曾擁有如此的集體出逃的頹廢。叛逆只是對落單的詛咒、是溼冷又落默的苔原氣候。逃,又能去何?無處可逃。而費洛蒙不會憑空消失,總須有處洩出。身體得多,力行卻少。力量在一個個對抗假想敵的模擬器中消磨殆盡,讓我們的氣息微弱、生長出充斥著惰性的肉瘤{——}要麼服從社會規則,要麼扭曲妳的思維使妳進行無窮盡的錯誤選擇{——}像是貼著無力掙脫的撒隆巴斯,治療著怎樣都好不了的抑鬱。空氣都在蒸發。集體頹廢成為不了我們的未來。

未來是什麼呢?一個有趣的解答是「\href{https://www.behance.net/gallery/88971815/Pause-Originals-Cyberpunk-Shaihaiching}{賽博龐克山海經}」。影片裡古典的霓虹與螢幕同柔面反光的黑暗質感相合,強調肅穆典雅和雋永的機感美、允許弱勢族群的頹廢和進行改造的執生欲,似乎是集體頹廢的反面。如今這種雅和欲,在黨國高層和朝野底層的思維和美學裡面,正如對集體頹廢的默允,也是蕩然無存。

\vspace{3em}
\begin{flushleft}
\small{御薬袋托托\\
二〇二〇年二月廿二日}
\end{flushleft}
\end{document}