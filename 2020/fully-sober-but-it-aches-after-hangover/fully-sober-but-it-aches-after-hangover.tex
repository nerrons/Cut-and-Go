\documentclass[10pt]{article}

\usepackage[a4paper,
	top=28mm, right=60mm, bottom=40mm, left=20mm]{geometry}

\usepackage{fontspec, xeCJK}
\setmainfont[Ligatures={Rare, Historic}]{EB Garamond}
\setCJKmainfont{Source Han Serif TC}

\setlength{\parindent}{0em}
\setlength{\parskip}{0.8em}
\linespread{1.3}

\usepackage{xcolor}
\definecolor{toto}{HTML}{E8D08F}

\usepackage{hyperref}
\hypersetup{
    colorlinks=true,     
    urlcolor=gray
}

\providecommand{\tightlist}{\setlength{\itemsep}{0pt}\setlength{\parskip}{0pt}}

\begin{document}
\begin{flushleft}
\textbf{華彩}\\
Fully Sober, but It Aches After Hangover\\
\end{flushleft}
\vspace{3em}

今日在翻舊書,瞥見我以前唸不下書的撕裂的日子。在 John Milton
的一本小冊子《\href{https://en.wikipedia.org/wiki/Areopagitica}{A Speech
for the Liberty of Unlicensed
Printing}》(一六四四年)裡,一頁我貼了標籤。其中勾劃出 Milton 引述
Francis Bacon 的金句:

\begin{quote}
The punishing of wits enhances their authority, and a forbidden writing
is thought to be a certain spark of truth that flies up in illegible the
faces of them who seek to tread it out.
(譯:懲罰智識反增其威,禁錮寫作譬若一束真理火花滑過妄圖踐踏者之臉頰。)
\end{quote}

我不確定是否為 Bacon 的話,John Milton
自己撰的也說不定。無論如何這句話在幼時的體內埋下了種子。或臧或否,我下意識地相信它。轉念一想又藏著太多的不甘,和不安定感。

假設有這樣的孩子,身在某種壓抑的看護機制之中。她受到太多嚴厲的制約和控制,納入一種強烈的懲罰規則當中。而當看護、控制的一方對她之於整個世界的價值受到挑戰時{——}這種情形的產生可能源於洋派、新潮文化衝擊,起初一定會流連於其光影,當作一種模仿式的流行。從異域文化引向某一深刻、哲藝的非流行的面向。但是終於,這種文化衝擊進行到不能夠被本源文化常規化的時候{——}即意識到控制體的狹窄,反叛的做法滋生,她走入里徑。行至此處,有太多不需要假設的現實的例子,古今也不足奇。而奇事要事是:反叛的姿態在不斷積累,壓抑和恐慌使得她寸步難行。

事實上她是懦弱的、缺乏安全感,以及 love
disabled(不能夠愛上厭惡的東西,哪怕有血緣搭橋)。她足夠清醒,但是是宿醉後的清醒,疼痛遍及全身。這種不安定的墜落感體現在:

\begin{itemize}
\tightlist
\item
  她的疑惑總是「什麼不是(地理和時間的)當下?」對當下過敏。
\item
  因為跳脫當下,會有一根迴路將本源文化視作假想敵。
\item
  因為存在假想敵,未辨敵友時,遊說的渴望和地下社交恐懼症又產生矛盾。
\item
  對疼痛上癮。
\end{itemize}

那麼她可能犯的謬誤:

\begin{itemize}
\tightlist
\item
  因為文化默認輸入方式的緣故,會把美學判斷當成是安全感評估。
\item
  安全感的建立一部分來源於情感,而有一部分曾經與壓迫者分享而如今多少被碾碎剔除,需要更多情感的成分去補充。她是忽冷忽熱的怪獸。
\end{itemize}

我寫作的母題即是這種〇〇一代的宿醉,藉此勉勵陷入某種囹圄的各位。Now
we're fully sober,but it aches after hangover。

\vspace{3em}
\begin{flushleft}
\small{御薬袋托托\\
二〇二〇年六月十七日}
\end{flushleft}
\end{document}