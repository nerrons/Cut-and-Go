\documentclass[10pt]{article}

\usepackage[a4paper,
	top=28mm, right=60mm, bottom=40mm, left=20mm]{geometry}

\usepackage{fontspec, xeCJK}
\setmainfont[Ligatures={Rare, Historic}]{EB Garamond}
\setCJKmainfont{Source Han Serif TC}

\setlength{\parindent}{0em}
\setlength{\parskip}{0.8em}
\linespread{1.3}

\usepackage{xcolor}
\definecolor{toto}{HTML}{E8D08F}

\usepackage{hyperref}
\hypersetup{
    colorlinks=true,     
    urlcolor=gray
}

\providecommand{\tightlist}{\setlength{\itemsep}{0pt}\setlength{\parskip}{0pt}}

\begin{document}
\begin{flushleft}
\textbf{Another Catalogue of Artists from China}\\
\end{flushleft}
\vspace{3em}

\textbf{太容易出戲}

一月恰巧在獅城烏節路上聽過 \href{https://typesettingsg.com}{Yao Yu Sun
先生}關於在地傳統鉛字史的講座。幫忙照應秩序與器材的有位梳 high alfo puff
的非裔工作人員,她著的深藍半袖制服流露出健美曲線,卻有東方人特有的丹鳳眼與溫潤氣質,從周遭的亞細亞面龐之中獨立開來。我以外人的姿態坐在區域靠後的角落,視線遮掩著越過人群隔隙,間斷地向她那頭偷望去,也碰觸到她的目光。又是害怕,心裡面卻總是蕩漾。

講座的第二部分 Yao 先生指導大家在 Pro­cre­ate
上繪字。不知是他作為藝術家的某種雷達還是根據我相貌生發出來的預設,下巡時先生在身旁責問我是否是個
cre­at­ive,我戰戰兢兢地講否。這才意識到大家將這個活動僅僅當作午後消遣。諷刺的是,同樣性質的講座在我的國家裡面從來都不曾涉及如此主題,這令我感到沮喪。

行至尾聲。假以我當下更清晰的頭腦,我也無法分辨她是讚好我的作品還是我們之間視線交換的原因,她走過來邀請我分享上螢。而滿是
geek
態,身形瘦小、手足魯鈍的我因為過分羞赧,英文在嘴邊呢喃。她也自然毫不客氣地讓我將作品
AirDrop
過去。此番不客氣無任何輕蔑的成分反過來是她的一種慢熱的傲骨。這牽動起我的某些陰霾記憶,然而我卻並未受其煩擾。在「醉醒狀態」下獨行踽踽的我,因國籍與文化的漫遊錯亂,渾身散發出在哪裡都不像是在地人的無所適從感。做事談愛是混蛋一氣,行文是混蛋風骨。看著自己的皮膚,又太容易出戲。

\vspace{1em}
\textbf{賽博唐人街}

說起肉空間的唐人街,除了種族隔離的歷史,首個跳入腦海裡面的是一九七四年由
\href{https://en.wikipedia.org/wiki/Roman_Polanski}{Ro­man Po­lanski}
執導的黑色電影《\href{https://en.wikipedia.org/wiki/Chinatown_(1974_film)}{Chin­atown}》(時間線:一九六九年
Polanski 的妻子 \href{https://en.wikipedia.org/wiki/Sharon_Tate}{Shar­on
Tate}
與腹中的孩子在洛杉磯的住宅被殺)。此電影邪門的一點是,故事的舞台大多不在唐人街。而唐人街則更像是一種被矮化黑化的、異域情調的隱喻,是法外之地的代名詞。之後的電影也部份繼承了這樣的印象。而近來有許多闢蹊徑的刻畫,像是攝影師
Jay­son Pala­cio
\href{https://hypebeast.com/2018/2/gosha-rubchinskiy-brain-dead-blends-editorial}{在洛杉磯唐人街的作品},充斥著叛離的味道和霓虹的賽博感。

前人的敘述,李歐梵在一九七五年的文章《美國的「中國城」》,實在精彩:

\begin{quote}
從紐約地下車道的出口走出來,首先呈現在眼中的中國城是一片零亂:鮮紅色的電話亭,五花八門的店鋪招牌,歪歪斜斜的中國字,街道旁邊的小販,擁擠的行人,還有那一家門面已舊的
blue
電影院{——}外國「美女」的照片,配上不倫不類的中文譯名。一個初到紐約的遊客,一定會覺得這是一種「半下流社會」。我初到紐約的時候,朋友帶我去逛唐人街,我總覺得不舒服,看到那些年老的華僑那副「不務正業」的樣子,更是不順眼。在館子裡吃飯,不會用廣東話叫菜,被迫用英文,看堂倌的那一臉不屑的冷漠表情,真想拔腿就跑,不再受騙了,走到街上,又怕小偷或扒手,或街角上的不良少年。紐約的治安本來不好,唐人街又在義大利區旁邊,我對於義大利「Ma­fia」(黑社會)的恐懼,也帶到「唐人」身上了。⋯⋯

初時我對於自己的這種態度頗感不安,因為對中外電影,我顯然是用了兩種尺度,但經反覆思考之後,也覺得這種「畸形」的態度未可厚非,中國人在外國漂泊,在美國的「中國城」裡懷緬中國文化,本來也是帶有一點「畸形」的心理。我們這些留學生,本來就是「中國城」中的過客,我們在唐人街沒有根,而只是生活在唐人街的邊緣而已,而唐人街卻又在美國社會的邊緣,雙層隔膜之下,「中國城」豈不正像一部電影?而在「中國城」中看中國電影,更談不上文化上的「真實」了。⋯⋯

五六年前,我在舊金山的唐人街遇到一個怪人,他留了一頭長髮,在頭後盤了一個辮子,經過友人介紹以後,我問他對於海外中國文化的看法,不料他卻把辮子一揮,滔滔不絕地說:「什麼中國文化?你們這些留學生滿腦子就是中國文化,其實在美國社會哪裡有中國文化?我是在唐人街出生長大的,我不會說國語,我的母語是英文,我的國籍是美國,我的文化背景就是這又髒又亂、為白人所恥笑的唐人街!我現在要以唐人街為榮,把唐人街的真相用戲劇的方式表現出來,我不像『新聞週刊』中的那位華人記者,他根本是『白化』(white-washed)了,他哪裡能代表唐人街?我現在剛寫好一個劇本,正在排演,就是在諷刺他,也諷刺白人!」
\end{quote}

這些天朋友在談論一個叫 \href{https://twitter.com/hashtag/ACAC}{\#ACAC}
的標籤。\href{https://twitter.com/ACACsince5859}{A Chi­nese Art­ists
Cat­a­logue},意在跳脫中文互聯網的惡劣環境,為畫師提供互相扶持的他處。這是一場自救運動{——}他們在派發游泳圈。ACAC
講自己處於「賽博唐人街」。我不清楚是借用了之前的隱喻,還是僅僅在描述一種民族性聚居的共同體。

而希望雜居的朋友,搬出賽博唐人街吧。We want an­oth­or cat­a­logue for
Chi­nese
artists。同時忘不了的是\href{https://zh.wikipedia.org/wiki/\%E7\%89\%9B\%E8\%BB\%8A\%E6\%B0\%B4}{牛車水}的咖哩飯和雞雜麵。

\vspace{3em}
\begin{flushleft}
\small{御薬袋托托\\
二〇二〇年三月十三日}
\end{flushleft}
\end{document}