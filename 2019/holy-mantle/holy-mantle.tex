\documentclass[10pt]{article}

\usepackage[a4paper,
	top=28mm, right=60mm, bottom=40mm, left=20mm]{geometry}

\usepackage{fontspec, xeCJK}
\setmainfont[Ligatures={Rare, Historic}]{EB Garamond}
\setCJKmainfont{Source Han Serif TC}

\setlength{\parindent}{0em}
\setlength{\parskip}{0.8em}
\linespread{1.3}

\usepackage{xcolor}
\definecolor{toto}{HTML}{E8D08F}

\usepackage{hyperref}
\hypersetup{
    colorlinks=true,     
    urlcolor=gray
}

\providecommand{\tightlist}{\setlength{\itemsep}{0pt}\setlength{\parskip}{0pt}}

\begin{document}
\begin{flushleft}
\textbf{Holy Mantle}\\
\end{flushleft}
\vspace{3em}

近來好容易被激怒,不知道什麼時候會爆發,對親近的人也是這樣,因此不太敢講話。可以想像出來我如果怒出聲音來的樣子,隨時可能發生,覺得太可怖。這種怒氣不是具體針對誰誰誰的。爸爸在小時候會當著我的面怒罵讓他感到不舒服的陌生人,聲音超大、言語超暴力。這一點上我似乎有朝他。有些人在體內流淌的軟弱的顆粒,只有怒氣能夠撕開她趾高氣昂的皮囊,敲碎她鬆軟的骨頭。我憎惡對逐漸下沉的周遭無動於衷的人。我太討厭「沒辦法這地兒就是這樣子」之類的話。因為你的存在這地兒才會就這樣子。每次把「通俗」淪為解藥的話頭,我都遺憾地直接截停。賦權於我的話,我會是個浪漫的暴君吧。

不過我有在和別人談話。聽到有對自我主義特立獨行的、卻是引以為傲的
confess。她說集會能夠不參加就不參加,寧可自己做事情。一方面帶有脫離的傲氣和自信,一方面被別人
back talk
同時被嚮往。\textbf{做什麼事情都要取得特權}{——}這種特權不是因為值得,而是因為面帶羞容的弱小、或是受到某種表面上無法講明的欺壓和非公正的後發行為{——}是大家共同持有的特徵,這就是「沒辦法這地兒就是這樣子」的原因。

My missus
第一次讓我思考一些人被保護的程度。我總是講我在進行錯誤的選擇,選擇對我來說是概率的事情。然而能夠被保護起來的其他人,選擇是一種人格設計上的考量,是勾勒出人生的節點。當然我也會有陰謀論去想像他們的保護殼像雞蛋一樣裂開的那一天,流出來的是生蛋清。事實上我的邏輯讓我也生長在名為寫作的保護殼裡面,但是我的保護殼是帶有毒性的。想知道真相嗎?如果真相是現實的同義詞,那麼還是不知道為好。寫作創造出來的真相,被庸俗者唾棄,因為這不是他們所理解的現實;斯真相被弒君握在手裡面擠碎,出汁。

昨晚和 my missus
看了剛在國內上映的{坂}本龍一的\href{https://www.imdb.com/title/tt6578572/}{紀錄片}。(在劇場裡面有把它當爆米花電影或者是喜劇片看的。)我不禁覺得他的保護殼是多麼完整。聽《異步(\href{https://en.wikipedia.org/wiki/Async}{\emph{async}},二〇一七)》的時候就覺得〈fullmoon〉裡「你會想起多少次童年中/某個特定的下午」的意境好美。像是把黃瓜片塞進啤酒裡面。這番話原自
\href{https://en.wikiquote.org/wiki/Paul_Bowles}{Paul Bowles}
在一九四九年寫的《\href{https://en.wikipedia.org/wiki/The_Sheltering_Sky}{遮蔽的天空}(\emph{The
Sheltering
Sky})》,講的是可能性帶來的對無窮的錯視。這就是因無窮而作業的有毒性的保護殼,由於意識到有窮而消解了毒性。精力有窮,敵人有窮,靈感有窮,怒氣亦有窮。托托是披著
holy mantle 的托托。

\vspace{3em}
\begin{flushleft}
\small{御薬袋托托\\
二〇一九年十二月十九日}
\end{flushleft}
\end{document}