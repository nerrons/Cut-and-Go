\documentclass[10pt]{article}

\usepackage[a4paper,
	top=28mm, right=60mm, bottom=40mm, left=20mm]{geometry}

\usepackage{fontspec, xeCJK}
\setmainfont[Ligatures={Rare, Historic}]{EB Garamond}
\setCJKmainfont{Source Han Serif TC}

\setlength{\parindent}{0em}
\setlength{\parskip}{0.8em}
\linespread{1.3}

\usepackage{xcolor}
\definecolor{toto}{HTML}{E8D08F}

\usepackage{hyperref}
\hypersetup{
    colorlinks=true,     
    urlcolor=gray
}

\providecommand{\tightlist}{\setlength{\itemsep}{0pt}\setlength{\parskip}{0pt}}

\usepackage{amssymb}

\begin{document}
\begin{flushleft}
\textbf{蹦迪的奶牛}\\
Cow Dancing Disco\\
\end{flushleft}
\vspace{3em}

農夫們把那些\href{https://msh.mosreg.ru/sobytiya/novosti-ministerstva/25-11-2019-10-07-55-na-podmoskovnoy-ferme-testirovali-vr-ochki-dlya-ko}{俄羅斯奶牛}趕入虛擬實境裡面,希望它們能夠愉快地產奶,真是十分現代的牛圈。用電力去模擬的夏日牧場,是奶牛在生理上總想待的地方,所以它們喜歡什麼,我們就餵給它們想吃的東西。奶牛有意識到受騙嗎?我們在乎嗎?大概是不在乎這些生靈的,我們從來不在乎
generator
們的感受。特別是如果將這一騙局維持下去,永遠沒有碎裂的一天,大概它們也會毫髮無傷,察覺不出異樣。我們在收穫的期待中勒緊圈養著活物的柵欄,然後
mind our own business。

以前我們剝奪奶牛的自由,密集地關押在狹長的房間。現在我們\textbf{以最大的程度放寬了它們的自由},卻剝奪了視覺。(即將剝奪知覺?)有一天我們將喝到這樣的牛奶{——}奶牛欣賞的,是偽造的自然;哺育著我們的,是偽造自然的乳汁。同樣的甘甜。

上個月在保利看了由\href{https://www.scmp.com/news/china/society/article/3018456/daring-new-take-classic-chinese-play-teahouse-causes-stir-france}{孟京輝改編老舍的《茶館》},演員在舞臺上蹦迪。法國批評人講它太超過(over-the-top
special effects)和太低俗(naff stadium rock
opera)。中途不斷有人離場,亦有廣眾直接打斷表演質疑是否是在演茶館。起火點我想是當時在諷刺生活低潮的垃圾話,它直接戳在了觀眾的心口上{——}我$\blacksquare\blacksquare$不需要你來羞辱我的現實生活我給錢是要你餵給我茶館的混蛋。奶牛看見了讓它們恐懼的東西,之後它們受驚著怒吼。

當你發現其實自己才是奶牛的時候,do you fancy a cigarette out
there?出門透透氣,然後默默回到暖屋。必須要活下去。有那麼一些瞬間,你不大想把這種眩暈感概括為現世的焦慮。期望著有那麼一天,奶牛們掙破
VR 頭盔,一齊向圍欄外面衝。但這實在是另一種虛擬幻象。

\vspace{3em}
\begin{flushleft}
\small{御薬袋托托\\
二〇一九年十二月三日}
\end{flushleft}
\end{document}