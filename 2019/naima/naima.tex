\documentclass[10pt]{article}

\usepackage[a4paper,
	top=28mm, right=60mm, bottom=40mm, left=20mm]{geometry}

\usepackage{fontspec, xeCJK}
\setmainfont[Ligatures={Rare, Historic}]{EB Garamond}
\setCJKmainfont{Source Han Serif TC}

\setlength{\parindent}{0em}
\setlength{\parskip}{0.8em}
\linespread{1.3}

\usepackage{xcolor}
\definecolor{toto}{HTML}{E8D08F}

\usepackage{hyperref}
\hypersetup{
    colorlinks=true,     
    urlcolor=gray
}

\providecommand{\tightlist}{\setlength{\itemsep}{0pt}\setlength{\parskip}{0pt}}

\begin{document}
\begin{flushleft}
\textbf{奈瑪}\\
Naima\\
\end{flushleft}
\vspace{3em}

五九年五月 John Coltrane
在《\href{https://en.wikipedia.org/wiki/Giant_Steps}{Giant
Steps}》錄入的第六軌〈\href{https://en.wikipedia.org/wiki/Naima}{奈瑪(Naima)}〉一直是我的私好。它擁有著我在夏日傍晚灑滿了夕陽的回途巴士車廂裡——山城道路多顛簸——在迴環的下坡路扶著把手聽它的時間膠囊。這一軌是柯川送給第一任妻子
Juanita Austin
的情詩。那時是從未戀愛過,或許陷入戀愛也只是愛戀愛情幻象本身的我。柯川不只將〈奈瑪〉錄在了《Giant
Steps》裡面,它還在很多專輯裡面現身(讓 Alice Coltrane
彈奏過)。相比之下,標題歌〈Giant Steps〉本身卻再也沒有在別處重錄過。

音符不斷在七和絃周圍挪移。緩慢的色士風,包裹著獨寫的鋼琴樂段。柯川在筆記中講它外沿是降
E 調,內蕊是降 B
調,這首歌是自己不完全無望的嘆息。我聽出了將晴未晴的溽熱意味。自四八年他淪陷在海洛因與酒精之中。五三年他與奈瑪相戀。奈瑪信伊斯蘭教,柯川出身於天主教家庭。兩年之後他進入
Miles Davis
五重奏時,他們結婚。在此之後他們在紐約與費城之間輾轉。又過了兩年柯川因為宗教的洗禮決心從惡癮中掙脫,他把自己鎖進母親在費城的房子裡面,奈瑪照顧他。柯川不但想要掙脫毒物的泥澤,他也希望掙脫
Miles 的陰影。這一時期柯川的歌有向著宗教性和精神性的漸變。

〈奈瑪〉沭浴在留有空隙的憂鬱陽光下,參雜著某種辛辣的隱恨意味。它既是爛漫抒情的糖霜,也似乎預示著六〇年代的情變。它嘗試著討論一種可能性,這種可能性在六四年〈Wise
One〉中因為情緒的不調和而產生的哀愁之中得到了呼應。這種可能性,滋生了他對安定的厭倦,讓他在六三年夏日搬出了他們在
\href{https://en.wikipedia.org/wiki/St._Albans,_Queens}{St. Albans}
的屋子,僅留下對奈瑪的柔軟的片語隻言。

信念被摩滅,被其他的東西所填充。維持信念是至少兩個人的事情,若單單是響聲在胸中迴盪、踽踽獨行,未免也太冒失與無趣。把信念當作解藥和救贖,信念卻將你視作毒藥和累贅。唯一能夠做的善事是揭發自我的虛偽作為對他者的坦誠,這是當下迷離的態勢。我依然像第一次聽〈奈瑪〉那樣無饜。〈奈瑪〉的陽光能否消解這種現代的壓抑,又會發生什麼事情去預示裂縫呢?「願榮光歸香港」或者是\href{https://twitter.com/ElectionDay_bot}{二〇二〇的大選}?一〇年代的最後一年還剩下\href{https://twitter.com/year_progress/status/1202905524326608896}{七趴},窗外是枯樹和煙霧,但是世界像是在暗室裡面洗膠片,那樣
no light、no light。

\vspace{3em}
\begin{flushleft}
\small{御薬袋托托\\
二〇一九年十二月八日}
\end{flushleft}
\end{document}