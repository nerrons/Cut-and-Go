\documentclass[10pt]{article}

\usepackage[a4paper,
	top=28mm, right=60mm, bottom=40mm, left=20mm]{geometry}

\usepackage{fontspec, xeCJK}
\setmainfont[Ligatures={Rare, Historic}]{EB Garamond}
\setCJKmainfont{Source Han Serif TC}

\setlength{\parindent}{0em}
\setlength{\parskip}{0.8em}
\linespread{1.3}

\usepackage{xcolor}
\definecolor{toto}{HTML}{E8D08F}

\usepackage{hyperref}
\hypersetup{
    colorlinks=true,     
    urlcolor=gray
}

\providecommand{\tightlist}{\setlength{\itemsep}{0pt}\setlength{\parskip}{0pt}}

\begin{document}
\begin{flushleft}
\textbf{區域 A 和區域 B 的界線}\\
The Split Between\\
\end{flushleft}
\vspace{3em}

笑匠 \href{https://en.wikipedia.org/wiki/Colin_Quinn}{Colin Quinn}
在他的脫口秀節目\href{https://www.imdb.com/title/tt10403090/}{《Red
State Blue State》}中侃道:

\begin{quote}
God put gaint mountain ranges and rivers all over the place. We didn't
take the hint from God. That was to indicate different countries.
Europe, same size as us, basically. Bunch of different countries. 'Cause
they understand. Every 700 miles, people have a different personality. I
mean, do you really think Hungary and Scotland have less in common than
Utah and New Jersey?
(老天把山巒江河置於四方,我們卻不能夠理解,那是用來區分國家的。歐洲和美國一樣大,卻有很多國家。因為他們明白,每七百英里,人的性格就會有所不同。我說,大家真的覺得匈牙利和蘇格蘭比猶他州和新澤西州更不像嗎?)
\end{quote}

但是,區域 A 首府到區域 B
的距離也跟匈牙利到蘇格蘭差不多。僅僅是不同,卻不存在區域 A 比區域 B
在理想上優越的情況。雙方相互的指責是否以優越感為底,才是決定言論是否值得延展下去的前提。

在區域 A 和區域 B
之間有一條界線,這個界線無益過分強調。形成對「獨立」的描述,是在主動地去指示或者說是去肯定一種隔閡,隔閡因為這個描述而確確實實地存在。無論是去討論界線破裂的惡果,還是嘗試琢磨維繫它的行動,是站在何種基本點上?損人利己還是世界性的良知。民意有權被尊重和完全地展示。如果事件是在六月份就發生,八月份才開始討論會很奇怪吧。區域
A 裡面如果某類觀點獨當一面,從來沒有為區域 B
辯護的聲音,會更加荒誕吧。去脈絡化的民族情結,僅僅是加固了雙方之間無差別的怒和恨。

區域 A 不過有兩點訴求。

一是官僚主義的枯敗。理性訴求得到傳遞,自大的固有集團受到瓦解和更新;另為普世文藝的發展。很大程度上得藉助人的野性。有很瘋癲的程度在裡頭,也有很自由超脫的元素。在廿一世紀做出真正能夠打動人心的東西。

區域 A 是在與區域 B 對立還是,因為與區域 C
的對立處於下風,轉化壓力降於區域 B 的身上,這樣產生的對立。

\vspace{3em}
\begin{flushleft}
\small{御薬袋托托\\
二〇一九年八月廿四日}
\end{flushleft}
\end{document}