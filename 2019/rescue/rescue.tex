\documentclass[10pt]{article}

\usepackage[a4paper,
	top=28mm, right=60mm, bottom=40mm, left=20mm]{geometry}

\usepackage{fontspec, xeCJK}
\setmainfont[Ligatures={Rare, Historic}]{EB Garamond}
\setCJKmainfont{Source Han Serif TC}

\setlength{\parindent}{0em}
\setlength{\parskip}{0.8em}
\linespread{1.3}

\usepackage{xcolor}
\definecolor{toto}{HTML}{E8D08F}

\usepackage{hyperref}
\hypersetup{
    colorlinks=true,     
    urlcolor=gray
}

\providecommand{\tightlist}{\setlength{\itemsep}{0pt}\setlength{\parskip}{0pt}}

\begin{document}
\begin{flushleft}
\textbf{救援}\\
Rescue\\
\end{flushleft}
\vspace{3em}

這是翻譯自離線雜誌\href{https://thedisconnect.co/}{《The
Disconnect》}二〇一八年第一期的封面故事\href{https://thedisconnect.co/one/rescue}{《Rescue》},由
Brain Mihok 寫作並授權。
\vspace{1em}

早上湯姆想著要不要離開飛船墜毀的地方。他望著埋艾莫斯上校屍體的土堆。一把鏟子豎立於其上當作臨時的記號。似乎他處在某個沙漠中央,除了尋找棲身之處外別無選擇。還早著,天色漆黑一片。他望向天空,短促地閃爍著的光線,像是隕石般歇斯底里地從大氣層中疾馳而過,隨即炸裂。

\begin{center}❧\end{center}

湯姆沒有關掉頭盔上的指示燈。但是當他發出聲響、呼吸、吞嚥或是低聲自語時,都會聽見接收器傳來類似於電話迴響的聲音。他也不願意冒險摘下頭盔。在不得不暴露於充滿微生物、真菌和蟲子的環境之前,頭盔能起到保護的作用。如果將供氧系統調至最低流速,它還可以維持八個小時的運作。也許活不到那個時候。艙內還保留有再次將氧氣柱填滿的裝置,那是在不得不呼吸這個星球上的空氣的時候。

\begin{center}❧\end{center}

遇險訊息在宇宙中遊動,朝著家的方向挪移。他沒有依照協議發送它,但是只要停下來想想大家接收到訊息的可能,他也就不感到那麼壓抑了。他清點了工具和補給品。接著他默默記住艙內的材料、剩餘的氧氣、工具箱、乾糧和水,最終決定離開這個地方。講真,離開的原因很大程度上與即將到來的無趣——他要在補給品有限的情況下劫後餘生——有關。他知道應該儘量避免無趣。空中的光再次閃爍起來,他瞇著眼睛,嘗試著找到更多的細節,更多的能夠徹底解釋他究竟在哪裡的證據。

\begin{center}❧\end{center}

他看不見蟲子也聽不見鳥鳴。沒有生物在沙中活動的跡象,找不著任何的排洩物。除了一種與岩石顏色相近的野草——野草的沙褐色讓他想起了家中叔叔的田地,那裡有長得極高的雜草——之外,沒有別的植物了。眼前岩石聳立,土質沙化,平坦的漂礫組成岩架從地上伸展開來,形成了高地。一個方向上是無邊無際的平地,另一個方向被眼前醒目的山岩阻遏了去路。他的前方似乎有一片山。

\begin{center}❧\end{center}

背靠在山脊的岩壁上,他發現了一排像是倒立的拖把布那樣的植物。他把石頭握在手中,戳了一下垂下來保護枝幹的樹葉。接著他撥開樹葉,果實就從中裸露了出來——拳頭大小的鱗莖狀物,有著和野草一樣的棕色。他摘下一個。殼摸起來不怎麼硬,鱷梨的形狀,有許多從上至下的縫隙。搖晃起來,汁液在裡面流動,發出聲響。扳開它、擠出來滴入檢測儀的隔層中。成分分析得花上一個小時。熒幕顯示液體的酸鹼度為七。他坐下來,雙手交叉,搭在他的膝蓋上。他意識到在這裡等待大概是一件需要適應的事情。

\begin{center}❧\end{center}

低矮的山巒是幾百萬年前地殼運動的產物,幾近垂直地從地上豎立起來,因此難以攀登。突出的漂礫之間自然形成了一條通向內部的小徑。湯姆轉過身,看著剛剛經過的平原,在後面,懸掛著突出的野果,山岩遮擋了地平線上的太空艙。他一邊深呼吸一邊走進這條小徑,這時響起了三種調性的嘟嘟聲,提醒他氧氣還有三十分鐘的餘量。他停下了腳步。想起卡德拉、父母、還是試飛員的那些年,以及攻讀學位的那些夜晚。他作出的每一個決定都達到了那些他從未刻意去爭取的目標。無論他是否意識到目標的存在,他做的事情都與別人不同。在別人都想來點兒刺激的時候,他一心求安;而無人想要接下某個危險任務時,他的名字總是最先出現在名單裡。假若其他人都不願意務農,他就會去。不過有太多務農的人了,也有太多開飛機的人了,而只有很少的人會願意同艾莫斯上校一道搭乘火箭鑽進深空,進行幾乎是有去無回的任務:盡可能地拓寬人類的邊界,並送回探索到的一切信息,這是人類站在邊界內的共感。

\begin{center}❧\end{center}

他想知道現在地球上的人們正在做什麼。地球還是在繞著地軸旋轉,家人已經睡下了吧。這個時候卡德拉也許在埋頭趕她的畢業論文吧,他想。過去好些年了,她也許將要搬到一個新的地方,開始新的生活。她曾經說過,執行任務的時候會想他的。思緒觸及此處,似乎在彌補失去情人的疼痛。無論如何湯姆是屬於數據的人,是屬於模式和重複的人。他只要沈浸在某種力場中,就會渾然不覺身邊的人正離他遠去。此次任務的訓練曾將他隔離了數月。現在他從任務中活了下來。任務失敗了,他想。接著他抬起頭看向一塊岩石,一塊與週遭環境格格不入的,有小客車那麼大的岩石。這塊岩石在一個無人問津的星球上,他想,直到我的來到。像是在很久以前的西伯利亞部落,穿過大陸橋來到一片新的大陸,他也像這樣繼續前進,因為眼前總是不斷浮現出新的東西,因為他是一顆從莢子裡鬆動的種子,隨風飄拂。

\begin{center}❧\end{center}

一聲嘶聲後,氧氣柱的封口機損壞,他立刻嗅到了硫的味道。外面的空氣暖暖的,他深吸一口氣,讓氧氣充入他的肺中。幾分鐘之後,他注意到有什麼物質在鼻腔中積聚。這使得他費力地噴鼻息、咳嗽,能咳出來多少咳多少。毒辣的太陽迫使他舉起雙手,保護他的眼睛。太亮堂了,他找不著方向。恍然之間像是來到了新墨西哥的某個地方。這個地方的影像在腦海裡變得毫無頭緒,隨著時間的緩緩流逝,影像也變得愈加雜亂無章。在他深呼吸的時候,他發現自己已經來到了這裡。

\begin{center}❧\end{center}

前方的空地中央顯露出一個巨大的黑湖。黑湖在下面還有些遠的地方,他還得向下走。硫的氣味變得濃郁起來,他不時地咳喘,好把它排出來。湖水,如果可以稱得上是水的話,輕柔地拍打著岸邊的細沙。湯姆拾起一塊石頭,拋向湖中,驚擾了湖水,泛起陣陣漣漪。他再次取出檢測儀,之前野果的汁液已經蒸乾,隔層也消了毒,他把檢測儀蘸入湖水中。酸鹼度為四。在示數下面有上次操作結果的簡報:野果的厚皮裡面的液體,帶有一些常見的微量元素,可以食用。

\begin{center}❧\end{center}

夜空佈滿星辰,輪廓微微泛著粉紅。他靠在一塊岩石上,喫著口糧,這是一種密實的、麵包似的塊狀物,有著柔和的、微微鹹口的,像是薄脆餅乾和玉米粉糕混合在一起的味道。他取出儲藏在制服內襯的水囊,抿了一口。\textbf{有人會來救我嗎?}他想。其實不必問,之前經歷的訓練已經告訴了他此類問題的答案。

\begin{center}❧\end{center}

太陽比在地球上看到的還要大,直射的時候酷熱難耐。如果在太陽底下待太久,他會被眼前的景物催眠,冥冥之中像有什麼東西指引著他四處遊走。媽媽站在遠處的小丘上。頭頂懸著一個黑色的球。他擔心他的書籍會突然失蹤,他希望有人來熄滅太陽。風不時地吹來,有駐唱在這裡面的樂手。他認識的所有人都在大桌子旁享用著豐盛的餐食。唯獨他落單了。到處都是腐爛的雞蛋。乾燥的地面使得他每走一步都會發出摩擦的聲響。在上方的山中有好幾處洞穴。他走上去進入了其中的一個,躺下來休息。卡德拉正想給他講,那些一直沒來得及說的話的時候啊,他已經沈沈睡去。

\begin{center}❧\end{center}

熒光棒的光芒探入漆黑的洞穴中。洞穴逐漸變窄,不像是古老的河或是海沖刷而成的,而像是厚重的岩石板跌落下來,搖搖欲墜地相互倚靠在這裡。向裡面走,延伸了很遠,直到上下相接,已達極裡。湯姆用熒光棒照亮了一片牆,一個圓圈浮現出來。直徑至少有兩米長,像是雕刻或是繪製在上面。當他用手觸摸它時,並沒有把圓圈刮落下來。\textbf{由什麼製成的呢?}他想。甚至也問起自己,\textbf{是誰}?他回到洞穴的入口,再次坐下,望著落日灼烤大地。

\begin{center}❧\end{center}

他意識到即使在夜晚將息,萬籟俱寂之時,天空也沒有呈現出單一的顏色。太多星星了,星星之外的太空幾乎模糊不清。若他不聚焦於此,把自己的感知全然交給周圍,可以體會到坡度的變化。綠色的藍色的灰色的陰影塗抹在一塊兒,像是一團巨大的暴風雨雲團透著零星的光斑和縫隙。他把這些顏色與地球的夜空藍色對比。腳下的地面堅硬,不適合就地躺下,所以他匍匐進入一塊沙地。汗濕透了背心。他睡到了早晨,太陽尚未升起,空氣依舊沈悶。寂靜無聲。

\begin{center}❧\end{center}

他一步步站穩,把雙手掛在岩壁上。手指懸垂在岩壁的縫隙中,全身懸空。他變換重心,不斷挪移,汗珠滑過他的臉龐。他的下面有幾塊岩石,距離不遠。他放開手,跌落在腳底下的岩架。臉蹭到了沙子裡面。天空綠藍相容,掌管著一切。他漸漸恢復了呼吸,下面沒有可以落腳的岩架,二十來尺的岩壁延伸到地面。他吸入裝備在制服內襯的壓縮水。他的皮膚被灼傷,腦袋裡面感到一陣暈厥。跪下來,他再次將自己懸在岩架上,由此降低下落的高度。之後他鬆開了手。

\begin{center}❧\end{center}

在制服中的水囊耗盡之後,他把另一邊的拉鍊拉開,用一些杆子支撐起來,搭成一個臨時的帳篷。他把器械包關上,這樣可以避免太陽直曬。黃昏之時他去找野果。好幾週他都在喝下果汁之前進行了測試,但最終他還是漸漸厭倦了這麼講究,他很難還像是剛墜毀時那樣進行細緻的檢驗。一天,汗流浹背的他決定冒險回到太空艙那邊去拿一些有用的材料,然後搭建起一個更加牢固、更加安全的防護結構。他剛要出發時,雲層聚攏,愈積愈高,天空陰沈下來。他以最快的速度跑上小丘。他還沒有經歷過暴風雨,風也許會很危險,也許會有電閃雷鳴。天知道會降下來什麼。

\begin{center}❧\end{center}

向東走,土地越加貧瘠。西方,在一片低矮的高原的那頭,幾座不可徒手勘測的山擋住了去路。南方層次分明,山岩崎嶇,路從中生。看來很久以前有河流沖刷出一條蜿蜒的小徑,它同時向赤道兩側延伸,他對辨認哪裡算得上是熱帶並不感興趣。北方是小一點兒的山脈,黑湖就在那裡,長滿沙褐色雜草和沙質平原也在那裡。幾週以來他與太空艙保持近距離,並且通過在岩壁上刮出的橫線來確定位置,不至於偏離。那時他已拆下座艙內部的面板,繫在了棚屋屋頂頂。他扯下駕駛艙牆面上的薄膜覆在面板周圍,用來阻擋風沙。他把艙內的儀器、顯示器、支架等金屬組件綁在岩石塊上製成了刀具。他撕下由高密度紡織品製成的防輻射膜當作太陽光的屏障。暴露的太陽光給他的皮膚陣帶來陣發痛。他有一個簡單的計劃:不計一切代價找到食物來源。

\begin{center}❧\end{center}

他停下喃喃自語,開始大聲唱歌。有時當他靠近岩壁,聲音會像黏著皮筋一樣彈回。又過了一會兒,他也不再唱了。因為難以掩抑的悲傷,在回聲裡面迎面而來。

\begin{center}❧\end{center}

星球上有兩顆衛星,相距很遠。近的那一個的形狀與月球很像。遠的哪一個只有其三分之一。它們的園缺變化不同步,因此不會同時出現新月。這意味著不會出現完全漆黑一片的情況,湯姆對此非常感激。他看著遠處的群山,收束成地平線上的一條破碎的黑色直線。他看著充盈的天空,接著看看黑暗中的手,最後閉上了眼睛,念著家人的臉龐,他意識到可能再也見不到了。見不到辦公室,見不到週五午後在德比馬賽前臺坐著的朋友們,甚至見不到他現在站著的這個地方,他在黑夜中進入了夢鄉。

\begin{center}❧\end{center}

野果的外殼烹飪後可以食用。在檢測完樣本之後,他認識到野果中含有微溶的纖維,同地球上的有車前子和漦的果實,或是玉米羹很像。生吃的時候難嚼也難嚥,它們會流出粘稠的苦味;烹飪後變軟,苦味變成了單調,這是他願意品嚐的味道。他粗略認為這些果實可以食用,但之後他逐漸開始缺乏維他命B。他閉著眼睛坐下來,背靠在一塊漂礫上。腸道一陣一陣的痛,像是在身體的深處抽筋,把在這個與地球大小相近的星球上獨處的思考拋之腦後。在這個星球上只有他一個有知覺的存在,只有一個大腦正在經歷這裡的一切。他集中注意考慮他的身體的協調。他的胃不住疼痛,他來到漂礫的另一邊,脫下褲子排便。像是有個小人用鋸齒狀的勺子從他的身體裡刮出什麼東西。太陽漸漸落下,月光照耀大地。他想起了太陽的光芒。

他還記得東邊日出西邊雨的情形。他問媽媽為什麼。「雨從哪兒來的呢?」媽媽反問道。湯姆抬頭望向天空,又望望媽媽,認為雨從雲中來,但也拿不準。「太陽總在發光,」她說:「只不過有時雲擋住了光線,有時沒有,但太陽總是在發光。」

湯姆蹲下來,緩和身體並做了清潔。他回到漂礫的另一邊,躺下來進入了睡夢中,夢中有一望無際的草坪。成千上萬的草柔和地將他托舉起來,他甚至在夢中入睡。他在夢中夢裡夢到一個遙遠的地方,滿是岩石、沙漠和孤獨。當他醒來的時候,他竟身在夢中夢的地方——兩個衛星高掛在空中,地平線上一片寂靜,他的思緒沈溺在孤獨中。空氣愈加清爽,預示著一個更加和煦的季節即將到來。

\begin{center}❧\end{center}

在翻越一座小丘的時候,他看見一塊有突起岩石的土地。它們看起來像田裡的莊稼那樣排成了一列。花費比他想象的還要長的時間從小丘上下來,他意識到這些突起的石塔要更高大一些——大概有五米那麼高。這讓他聯想起復活節島、巨石陣、金字塔。在每一塊石塔之間都有足夠他穿過的空隙。比他還高上那麼一尺的岩石塔上,一個圓圈雕刻於上。他四處觀察的模樣像被囚禁在一個實驗裡面,他是實驗觀測的對象。而他什麼也沒有看見,只有來時的小丘、天空、太陽。沒有回應的聲音。除了他的獨處之外沒有其他的東西了。這個星球像是一個空洞的生命獨體。他的大腦是這裡唯一有精神性的東西,他看看這些石板和圓圈,它們告訴他此地已然面目全非。

\begin{center}❧\end{center}

日子一天天過去,如今他再也不能判斷是否又跨過一個月,是否又來到下一個季節,他開始感到絕望了。他發現的那些可食用植物大多是野草的鱗莖根部,其中混雜著希望和悔恨的味道——所謂希望,源於儘管不豐足但至少還可以獨特地活下去的事實;所謂後悔,源於所有的生存選擇都綑繫在這種千篇一律的生活定態,全然與一切人類珍視的其他活動無關。

\begin{center}❧\end{center}

湯姆坐在岩架上,散漫無神地望著遠方,思考他的旅程。幾週前,他穿過一片遼闊的平原,像是穿過冰川之間的古老棧道,他在那裡不慎摔下,滾落到底部。他撞到腦袋,使得眼前霧濛濛的一片,擦傷的手臂淌著血;他記得有腐臭味的沼澤地,除了叮人小蟲——這是第一種除了植物之外的生命跡象——在耳邊嗡鳴,還有一團爛泥;他記得遙遠的乾燥的岩石,沙土景觀的山脈,他曾到過那裡;他記得太空艙。在那之前的一切都模糊不清了。他說服自己:在太空艙之前的一切記憶都不存在。他在墜毀地點,也就是在艾莫斯上校的旁邊出生。來到這顆星球之前的記憶都是一種錯覺,是為了延緩失智的錯覺。有趣的是,這種錯覺本身就是一種失智。他來到這個星球一年。或者他一直都在這裡。他再也不獨自歌唱,也再不說太多話了。有時候,當他要靠近堤壩上的植稈,撕下被他稱作艾莫斯果子的外殼的時候,他會漫無目的地哼起歌來。經常他會創作一段旋律,卻沒有意識到他正在發出噪音,不過有時候他也會發覺而停下來。沒有想去做但是卻做成了,這種情況總是讓他感到困惑。生存畢竟是一件毋須刻意的事情。

\begin{center}❧\end{center}

恐懼。在山丘間奔跑,他的手緊按身體一側,臉因疼痛而扭曲。天逐漸暗下來,星星在一片綠色中閃鑠。他跑過小山谷的時候,側面被刺傷,腦袋裡浮出許多粒子嘗試著讓身體感到舒適起來,他到達了一種極度亢奮的狀態。腎上腺素流經在他的心臟。眼角的景緻變得模糊起來。

他放慢腳步,疼痛變得更加尖銳了,他能夠感覺到血液正在浸濕臉他的大腿。一隻手被染成了暗紅,又換了一隻手按壓。他跑到樹叢中躺下,把零星的樹枝堆成一摞。他的嘴巴因疼痛而掙開,涎液在匆忙堆積樹枝時滴了下來。他無法停下來去思考躺在這裡是臧是否,可能會被那些人抓住。但他知道他必須對傷口進行灼燒處理。

他從枝幹間摘下小片乾樹葉,放在一起生火,從器械包中取出磁性棒和小刀片,把他們的邊緣拼接在一起,火星飛濺出來,跳躍到樹葉上,它們的邊緣逐漸熔化,接著染上了紅橙的顏色。他把樹棍和枝幹加入火堆,吹旺了火苗,然後脫下衣服檢查創口——一個豌豆粒打小的洞,正緩慢地流血。他從工具箱中取出火鉗放在火焰中進行消毒。他等待火鉗冷卻以免過熱熔化了他的皮膚,之後他把火鉗的頭部夾入身體一側的小洞。他在黑夜中叫出聲來。他把火鉗抽出結束疼痛。顫抖佈滿了軀幹和手臂。他突然感到無比寒冷,傷口像一顆敏感的牙齒那樣,冰水在其間流轉。

彈丸也好,彈片也好,要麼被溶解了,要麼在身體中失去了作用,他懷著這樣的想法和希望。抬頭看著空中的閃光,就像墜毀後的那晚的,也像之後許許多多夜晚中看到的同樣的閃光,什麼東西會從中流露出來,指導他如何選擇。只是小小的隕石,卻產生出一種慰藉。他把樹枝加入火中,然後把火鉗放在火上灼烤。當變得紅熱的時候,他深吸一口氣,把火鉗靠近側面身體時,他開始急促喘息。他叫出聲來,金屬在傷口週圍的皮膚中熔化。他把火鉗扔掉,用盡全身力氣按壓著傷口。他在那裡停下來許久,雙眼緊閉。燒灼開始褪去,而其他的一切也跟著漸漸褪去。天空的黑色掌管著他週遭的荒野。甚至火也黯淡了下來。從眼縫中向外看,他想抗議,去發出聲音,對著黑暗的世界怒吼,咒罵它偷走了唯一可以在寂靜中找到美的途徑,偷走了有氣息的地方。\textbf{站起來},他想。\textbf{我站起來了嗎?}他的身體感到一絲遲鈍,像是牙醫把麻醉藥推入了脊椎,他的身體像是一塊準備拔出的壞牙。\textbf{拔掉吧},他想。\textbf{拔出來呀,把它取走,我不想看見它。}他沈睡入夢。

好幾天他都不能移動。任何的拉伸或彎曲都會讓刺疼貫穿全身,為此他會稍微側身,但又不能過猛。在創口週圍的肌肉遭受重創,皮膚龜裂,發紫發痛。只要觸碰到它就會火辣辣地痛起來。他緊緊地背上背包,拿出幾株細小的根莖。幾乎不能生吃,但是他需要卡路里。他把鱗莖分成幾小塊兒,嚼呀嚼呀然後把它吐出來,植物的纖維太難吞嚥了。他又再次分成幾小塊兒,拿了很小的一塊兒再嚼呀嚼呀生硬地吞下。他滿足地噴鼻息,甚至有一些精神失常。成為一個深空宇航員的想法在當時看來多麼荒謬,就像意識到邊界會自然吞噬那些靠近它的人。\textbf{我應該到木星、木星的那些衛星上去,就像其他人那樣},他想。

\begin{center}❧\end{center}

艾莫斯上校在任何交流中都面不改色。如果湯姆或是隊裡的其他人講了個笑話,艾莫斯會勉強擠出一個笑容,嘴角上彎,作個表示。他自己從不開玩笑。艾莫斯遇到場面失控時,比如說大家大笑不止,或是爭個不停,或是離題太遠,他都會說「我們繼續吧」然後重整大家的情緒。「繼續什麼?」湯姆不該這樣問。「正方形。」艾莫斯上校不滿地說。當例會繼續進行,來到如何如何在軟件更新中處理複度向量時,湯姆仍想著正方形,想著能夠回到更簡單的時候。

他坐在小土堆上眺望著湖面,水平如鏡,映襯著天空的綠色。他想起上校的臉,當他們進入大氣層的時候,當電腦通知他們系統出故障,後背援助失效時。這是一種對警告的專注的表情。湯姆想知道這是不是上校最後的表情。湯姆在最後一刻會回溯到那個說不來話、眯著眼睛、哭哭啼啼的像是剛從子宮裡出來的新生兒。這次是他旅程的結束而不是開始。

\begin{center}❧\end{center}

現在湯姆的步伐變得短促了,有時會失去平衡。他不能再像以前那樣跳躍或是蹲下。在他的側面身體有一部分僵直。有限的運動範圍意味著他對任何探索都要格外小心。不過自從受傷之後他也沒有心情繼續探索。事實上可以用錯亂來向別人形容自己的感覺,如果還有人可以跟他交談的話。但這裡沒有別人。至少沒有不會傷害他的人。

那個時候,他蹲在一片樹林裡面。在與另外一片相離較遠的樹林中,他聽見了像是腳步聲或者是鋪灑石頭的瑟瑟聲。他起身來到一片空地裡,那裡有一個正在發出砰砰聲的機器,像是晒乾的玉米放在一壺油中。毫無警告地,他聽見什麼東西穿過空氣。這個突然而來的力量推他倒地,荊棘刺入了他的身體一側。現在他開始懷疑是不是誤解了一切,他的受傷到底是被刺傷還是被射傷的。

\begin{center}❧\end{center}

在月光的照耀下的一個晚上,他發現了一塊位於枯涸溪澗邊的岩石,白色的圓圈雕刻在上面。他在想這是不是那些射擊他的人的標誌,當然,如果真的有人這麼做了的話。他盯著岩石看,之後四處看看,思考著他是否正被人監視著。接著他走開了,裝作他沒有發現這塊石頭的樣子,像是它毫無價值那樣。他還是一瘸一拐地跛行,創口尚未痊癒,走路的時候像是有沙袋繫在了他的一側臀部那麼重。也很痛,但因為許多部分都在一天不同的時候開始疼痛,他也很難分清楚它們。雨開始輕飄飄地下,雲層快速聚攏,籠罩天空。儘管稀疏的頭髮溼漉漉地耷在眼前,他也依然直視著前方的平原,想象著遠處的存在。他停下腳步,這一刻在他的內心中,空無一物。他想知道探索的意義究竟是什麼,但沒有大聲問出來。是為了讀和寫,是為了器皿,這個問題,當被簡化時,卻可能包囊了一切,夢和記憶襲來,他要在這裡面尋找答案。

\begin{center}❧\end{center}

在長滿草的遼闊平原上,儘管他已經在這個方向行走了一個多星期了,遠處的山脈也沒有明顯地變大。食物快要耗盡,而他卻沒有發現其他可食用的植物。這裡的草,檢測儀顯示它富含營養,但他不知道應該如何攝入。在一次失敗的嘗試之後,他做到了將草用熱水煨軟做成羹,這樣就可以吞嚥和消化了。夜風呼嘯,協帶著一絲涼意,沒有木塊可以燃燒取暖,他望著黑夜直至入夢。他在一聲巨響中醒來。像是一隻大貓或是一頭熊發出來的聲音。接著到了早晨。他的皮膚開始起雞皮疙瘩,要離開的時候,他思考著腹中的食物。他也思考著艾莫斯上校和太空艙,像是在心中檢查它們是否仍然存在,以確保它們保持真實,或者說已經變得真實。他拿出一些草放入臨時的塑料碗。他按下工具的按鈕,把這些莖幹搗碎,加入少許水。混合物分解成糊狀物,像極了棕色和綠色的燕麥,他喫下了。嚐起來很苦,有葉綠素滲出的餘味。當他收拾好東西再次出發,他決定把這個稱作艾莫斯羹。

\begin{center}❧\end{center}

他朝向這個地方最近的一座山前進,將它命為艾莫斯山峰。地勢在上升時變得陡峭,他很長時間地望著山峰。他沒有想要攀登這座山。艾莫斯山峰高聳入雲,他想象著自己攀爬它的樣子。眼淚打溼了視線。失敗的必然淹沒了他。任務的失敗。生存的失敗。甚至是生命的失敗,他懷疑到。

\begin{center}❧\end{center}

在他身邊有小塊的漂礫和幾堆岩石,就像是墜毀地點的岩石那樣。這些是藍色的和黑色的。估計他沒有走過一千里也有幾百公里了。他在一堆岩石中穿梭,不為別的,只是想看看岩石滾落的樣子。他向後退,用盡全力把一塊石頭踢起來,石頭撞上漂礫,留下一條不對稱的刮痕。湯姆走過去撿起石頭來。他看看這個標記,接著抬頭看向艾莫斯山峰,在回頭看看他來時的路。\textbf{宇宙最好遵循一系列線性事件來探索的規律},他想。他拿起岩石朝向漂礫,按在上面並拖動,劃出一個圓圈。這條細線與之前他發現的像極了。太陽藏在淺綠色的雲層中。一切都毫無生機。他發現的記號是他自己作下的嗎?剛才是他在劃出一個圓圈嗎?光線是否真的進入了他的雙眼,告訴他眼前的一切?或者他什麼也看不見,眼前只是一片幻影。

\begin{center}❧\end{center}

起先上山的道路不大陡峭。過了三個晚上,他的艾莫斯羹幾乎耗盡,他停下來眺望遠方,小腿和大腿留下數道刮痕,他已不能像以前那樣通過岩架了。他更多地用胳膊支撐,把全身的重量都擔負在兩側的二頭肌上。百無聊賴時就畫圈圈。眺望使他能夠看清之前他穿過的遼闊平原。天空清朗,可以看到幾百公里外的東西。他想這顆星球,它本身難道不也是一個圈嗎?在家裡面,他解決問題從來都不是難事。和卡德拉分手、新生考試的錯誤修正,從最開始它就錯誤地把他引向深空。對於所有的事情,解決方案都是唯一的。湯姆,是屬於忍受的男人,也將一直忍受下去。時間會整理好所有的情緒。他意識到這是他對數學、統計學、物理學、天文學感興趣的原因。它們都是需要大量專注力的學科,是為了忍耐、學習和貢獻的學科,最基本的要求就是專注力。專注於長時間的事情,這樣世界就會改變,不需要任何額外的幫助。

\begin{center}❧\end{center}

他朝著艾莫斯山峰的頂部攀登。翻過最後一處傾斜的梯級就會來到山頂。他需要考慮下一個任務是什麼,給他的腦袋一個確定的目標。天色黯淡下來的時候,他正把自己拉入頂部,雙臂已經很疲憊了,他的腿也僵直了,雙眼睜開又閉上。他聽著夜晚、山巒的聲音,聽著這個星球輕微的哼鳴。墜毀的兩年來,他已經訓練出自己的耳朵,在雙眼閉上時,哪怕是最細微的聲音也要仔細聆聽。他的面前是在漸漸黑下去的太空中劃過的星星。這麼久過去了他依然好奇地盯著它們看。有些東西還沒有弄明白,他想。那鋪在他身後綿延無盡的土地啊,是他生命的路,是他磨難的路,卻被撕成碎片,無法辨認了。因此當他回頭望向它,視線所及之處是小鳥輕紗般的巢,是無盡的光。

\begin{center}❧\end{center}

有時在夜晚,他會像是在等待著什麼一樣凝視前方。或者他是在尋找著什麼?千絲萬縷的想法環繞在他身旁,很難分辨出它們什麼時候會靠岸、什麼時候會離港。其中的一些是平行展開的,但他也不清楚之前是不是也從同樣的想法中得出了同樣地結論。也許是吧。也許有很多次了把。他還有新的想法嗎?

有力的響聲在夜空中炸開。是一聲還是兩聲?無疑這是一種聲爆,意味著有什麼東西以聲音的速度衝進了大氣層。也許意味著,他不是在這個地方僅有的探索者了。他是探索這個星球的人中的一員。如果這還能夠被稱作是探索的話。他把手放在僵硬的身體側面,傷口已經癒合了。

他希望,或許,他的求救信號已送達,這是前來拯救他的聲音。他的任務最終還是成功了。他看見光線留痕,划向艾莫斯上校遇難的地方。他想知道是不是時間在自動地整理,這是他們的飛船,穿過大氣層。可能在很遠的地方,另一個他將要把他自己無助地拖出飛船。艾莫斯上校的脖子斷裂。這個星球是一個奇怪又冷漠的地方,另一個他可以用盡餘生去探索、去療傷、去思考、去觀察。

湯姆把頭低下睡覺,考慮著明天早上他會向著天空留痕的地方出發,因為天空展示出種種新的發展跡象,這使得他燃起一種與生俱來的衝動。事實上,他一直以來行動的目的變得清晰起來:調查並忍受調查的結果。其他事情都是次要的。唯有調查、唯有忍受,再調查、再忍受,調查、忍受。

\vspace{3em}
\begin{flushleft}
\small{御薬袋托托\\
二〇一九年八月七日}
\end{flushleft}
\end{document}